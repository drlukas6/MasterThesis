Ovaj rad bavio se mogućnostima i kvalitetom obrade slike Kartezijskim genetskim programiranjem (CGP).
U početku bavi se generalno genetskim programiranjem (GP), napredujući u preciznije teme.
Detaljno se prikazuje i objašnjava tehnika CGP-a, obrada slike i generiranje podataka zadane raspodjele.

Kao metoda obrade slike koristi se konvolucija.
Konvolucijske metode koriste se u kombinaciji sa CGP-om kroz dva primjera.
Micanje šuma i detekciju rubova.

U nastavku dotiće se generiranje podataka zadane raspodjele gdje se CGP-om, točnije, arhitekturom \emph{CGPAE} oponašao rad autoenkodera.

Konvolucijske metode pokazuju obećavajuće rezultate pri korištenju genetskog programiranja kao metode za stvaranje filtera za sliku.
Posebice uz korištenje CGP-a i prednosti koje nosi kao što su više izlaza te ručno kontroliranje arhitekture i složenosti modela.
Čak iz skup podataka koji bi se za \emph{neuronske mreže} smatrao manjim CGP uz provođenje algoritma na CPU jedinici relativno brzo konvergira rješenju.

Nažalost, generiranje podataka ukazuje na određene nedostatke.
Genetsko programiranje, time i CGP, ne iskorištava potencijal gradijentnog spusta te pokazuje slabe performanse pri problemima velikih dimenzija i u slučaju velikog broja hiperparametara koji se optimiziraju.
Također, rad predlaže metodu serijskog spajanja više CGP jedinki i algoritam evolucije, no prednosti istog nisu jasno vidljive te je potrebno daljnje testiranje.
Do sličnog zaključka dolazi i \cite{conv_gen_programming}.

U budućem radu interesantno bi bilo istražiti potencijal prijenosnog učenja (\emph{eng. Transfer Learning}) s CGP-jedinkama.
Na primjer, može li CGP koji je iznimno uspješan u micanju $5\%$ šuma s slike poslužiti kao početna točka za učenje CGP-a koji miče $\geq 50\%$ šuma.

Isto tako, smatram da je bitno povući jasnu liniju između problema koji su prikladni za rješavanje genetskim programiranjem i onih prikladnijih za druge tehnike kao što su na primjer neuronske mreže.
