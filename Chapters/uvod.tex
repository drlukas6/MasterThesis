Prva asocijacija na spomen napredne obrade slike je umjetna inteligencija i duboke neuronske mreže.
Uz to, tehnika konvolucije pokazala se daleko najprimjerenija za rješavanje problema. \\
Problem koji se javlja s rezultantnim modelima dubokih neuronskih mreža je što su gotovo uvijek nerazumljivi i rade kao \emph{crne kutije}.
Točnije, ljudima je teško razumjeti i objasniti kako je mreža došla do izlaza kojeg je stvorila. \\
Velika prednost modela nastalim genetskim algoritmom je razumljivost. \\
Genetsko programiranje se i ranije koristilo kao način obrade slike uz razlike kao što su (\cite{cgp_image_processing}):
\begin{itemize}
	\item jednostavnija arhitektura, npr. stablo
	\item manji skup funkcija
\end{itemize}

Ovaj rad baviti će se uporabom \emph{Kartezijskog genetskog programiranja} (\emph{CGP}), čija je arhitektura inspirirana dubokim neuronskim mrežama, u svrhu obrade i analize slike. \\
U nastavku rada, ukratko će se objasniti što je genetsko programiranje te kako radi.
Detaljno će se predstaviti \emph{CGP} kao tehnika genetskog programiranja, izraziti prednosti nad klasičnim metodama te predstaviti način na koji su implementirane konvolucijske tehnike u svrhu provedbe eksperimenata.
Predstaviti će se svi provedeni eksperimenti, prikazati će se i objasniti rezultati te usporediti s željenim.
Na kraju, okušati će se u korištenju \emph{CGP-a} kao metode izrade \emph{autoencoder-a} te će se ispitati ograničenja korištenja genetskog programiranja kao prikladne tehnike za generiranje uzoraka zadane raspodjele.

Cilj rada je dati čitatelju ideju o radu i performansama \emph{CGP-a} kroz provedene eksperimente te poslužiti kao inspiracija i početka točka za buduće eksperimente.
