Prvi problem koji sam definirao je problem rekonstrukcije slike s dodanim šumom.
Promatramo sliku, definiranu kao matrica $I$ veličine $w \times h$ gdje za svaki element $x$ na koordinatama $(i, j)$ vrijedi $x_{(i, j)} \in [0, 255]$ što predstavlja intentzitet boje od crne prema bijeloj. \\
Sintetički šum koji sam dodao vrste \emph{"Salt and pepper"} odnosno soli i papra nazvan je tako jer određeni postotak nasumičnih vrijednosti postavi na bijelu ili crnu boju (slika \ref{fig:salt_pepper_example}).

\begin{figure}
	\centering
	\includegraphics[width=0.5\linewidth]{Experiments/GrainRemoval/input_example.png}
	\caption{Primjerak fotografije s $5\%$ šuma soli i papra}
	\label{fig:salt_pepper_example}
\end{figure}

Postotak slike na koji sam odlučio primjeniti šum je $5\%$, slično kao i \cite{cgp_image_processing} i \cite{Sekanina2011}.
U nastavku rada također ću predstaviti rezultate na većem postotku šuma od $40\%$.

Cilj eksperimenta je evoluirati filter $f(x): [0, 255]^8 \rightarrow [0, 255]$ koristeći konvolucijske metode i CGP koji će od oštećene slike $D$ reproducirati novu $Y = f(D)$ što bližu neoštećenom originalu $I$.
$$
\min_{Y, i, j} |Y_{(i, j)} - I_{(i, j)}|
$$
Iz susjedstva svake vrijednosti želimo dobiti što precizniju promatranu vrijednost. \\
Polazišna pretpostavka je ta su susjedne vrijednosti na slici u međusobnoj korelaciji do određene mjere te mogu pomoći u zaključivanju originalne vrijednosti.
Problem koji se može javiti je da u promatranom susjedstvu također imamo vrijednost koja je šum što narušava pretpostavku korelacije.
Navedeni problem je zanemaren te je CGP-u ostavljeno kao problem koji treba riješiti bez ljudskog znanja.

Jezgra $\omega$ koju sam odlučio koristiti je veličine $3 \times 3$ koja promatra $Moore$-ovo susjedstvo (\cite{jakobovic}).
\[
	\omega_{(i, j)}
	=
	\begin{bmatrix}
		x_{(i - 1, j - 1)} && x_{(i - 1, j)} && x_{(i - 1, j + 1)}\\
		x_{(i, j - 1)} && && x_{(i, j + 1)}\\
		x_{(i + 1, j - 1)} && x_{(i + 1, j)} && x_{(i + 1, j + 1)}
	\end{bmatrix}
\]
Razmišljanje iza korištenja jezgre koja ne koristi središnju, promatranu vrijednost je što tu vrijednost upravo pokušavamo predvidjeti i njena vrijednost, posebice ako je šum ne smije imati utjecaja.
Detaljniji prikaz djelovanja i rezultata mooreove jezgre vidljiv je na ilustraciji \ref{fig:moore_example}

\begin{figure}
	\centering
	\includegraphics[width=0.8\linewidth]{Illustrations/moore.png}
	\caption{Primjer čitanja slike mooreovom jezgrom te transformacija iz preuzetog dijela slike u vektor vrijednosti koristivih CGP-u}
	\label{fig:moore_example}
\end{figure}

Za rješavanje problema potrebno je definirati prikladne uvjete zaustavljanja.
Kondicijska funkcija koju sam odabrao je temeljena na $L1$ pogrešci, odnosno
\begin{gather*}
err = \sum_{i=1}^{n} |y_{cgp} - y_{skup\ podataka}| \\
kondicija = \frac{1}{err}
\end{gather*}
gdje je $y_{cgp}$ vrijednost koju računa CGP a $y_{skup\ podataka}$ vrijednost koju želimo predvidjeti i biti joj što bliže.
Jezgrom konvolucije prolaziti će se cijelom slikom čitajući vrijednosti osam susjeda pravilom opisanim gore pretvarajući na kraju očitanu matricu u vektor.
Vektor se predaje CGP jedinki koja na temelju ulaza i operacija koje se izvode u skrivenim slojevima računa izlaz.
Konačno, ideja je minimizirati $L1$ pogrešku $err$ odnosno dobiti jedinku s najvećom kondicijom.

\subsubsection{Postavke}
Početne slike su u punoj veličini $256 \times 256$.
Na slike je programski dodan nasumičan šum na $5\%$ vrijednosti.
Zbog polinomijalnog rasta broja vrijednosti s rastom slike promatrani su podskupovi veličine $30 \times 30$.
Time nastaje skup podataka od $900$ vrijednosti.
Fotografije \ref{fig:sp_samples} prikazuju slike koje su se koristile u pojedinim fazama.

Svaki eksperiment bio je ponovljen 30 puta.
Odabrani algoritam je $1 + \lambda$, $\lambda = 4$.
Koristi se samo operator mutacije koji mutira 1 aktivni gen po iteraciji.
Dozvoljeno je najviše $500$ generacija odnosno $2000$ evaluacija.
Također, u slučaju pada greške ispod $0.005$ se postupak zaustavlja.
Tablica \ref{table:sp_function_set} prikazuje funkcije koje su se mogle koristiti u čvorovima.
Funkcije su odabrane na način da izlaz uvijek odgovara skupu $[0, 255$] iako je dozvoljeno manje odnosno veće vrijednosti vratiti na najbližu dozvoljenu rubnu vrijednost koristeći jednakost $f(x) = max(min(f(x), 255), 0)$.

\begin{table}
	\centering
	\begin{tabular}{||c c c||}
		\hline
		Adresa funkcija & Funkcija & Broj ulaza \\ [0.5ex]
		\hline \hline
		0 & $max$ & 8 \\
		1 & $min$ & 8 \\ 
		2 & $avg$ & 8 \\ 
		3 & $mean$ & 8 \\ 
		4 & $sum\ mod\ 256$ & 8 \\ 
		5 & $\sqrt{sum}$ & 8 \\ 
		6 & $max - min$ & 8 \\ [1ex]
		\hline
	\end{tabular}
	\caption{Funkcije korištene u postupku micanja šuma sa slike}
	\label{table:sp_function_set}
\end{table}

\begin{figure}
	\caption{Fotografije na koje će biti primjenjen šum korištene u trening, validacijskoj i testnoj fazi}
	\begin{subfigure}[t]{0.45\textwidth}
		\includegraphics[width=\textwidth]{Experiments/GrainRemoval/lena.png}
		\caption{Lena, fotografija na kojoj će se izvoditi trening i na temelju koje će se model korigirati}
		\label{fig:sp_train_sample}
	\end{subfigure}
	\begin{subfigure}[t]{0.45\textwidth}
		\includegraphics[width=\textwidth]{Experiments/GrainRemoval/cameraman.jpg}
		\caption{Kamerman, fotografija na temelju koje će se računati validacijska kondicija CGP-a. Na temelju ove fotografije model se ne korigira}
		\label{fig:sp_val_sample}
	\end{subfigure}
	\begin{subfigure}[t]{0.45\textwidth}
		\includegraphics[width=\textwidth]{Experiments/GrainRemoval/trg.jpg}
		\caption{Trg bana Jelačića, fotografija koja će se predati konačnom modelu u testnoj fazi imitirajući tako problem iz \emph{stvarnog svijeta}}
		\label{fig:sp_test_sample}
	\end{subfigure}
	\label{fig:sp_samples}
\end{figure}

\subsubsection{Rezultati}
Rezultati su prikazani na slici \ref{fig:sp_result_grid}.

\begin{figure}
	\caption{Fotografije iz trening, validacijske i testne faze prije i nakon micanja šuma CGP-om. Na dnu je također prikazana razlika dobivene slike i one željene.}
	\begin{subfigure}[t]{0.32\textwidth}
		\includegraphics[width=\textwidth]{Experiments/GrainRemoval/noisy_lena.png}
		\caption{Lena s $5\%$ šuma}
	\end{subfigure}
	\begin{subfigure}[t]{0.32\textwidth}
		\includegraphics[width=\textwidth]{Experiments/GrainRemoval/cameraman_noisy.png}
		\caption{Kamerman s $5\%$ šuma}
	\end{subfigure}
	\begin{subfigure}[t]{0.32\textwidth}
		\includegraphics[width=\textwidth]{Experiments/GrainRemoval/trg_noisy.png}
		\caption{Testna slika s $5\%$ šuma}
	\end{subfigure}
	\begin{subfigure}[t]{0.32\textwidth}
		\includegraphics[width=\textwidth]{Experiments/GrainRemoval/sp_output.png}
		\caption{Lena nakon prolaska kroz dobiveni filter}
	\end{subfigure}
	\begin{subfigure}[t]{0.32\textwidth}
		\includegraphics[width=\textwidth]{Experiments/GrainRemoval/cameraman_out.png}
		\caption{Kamerman nakon prolaska kroz dobiveni filter}
	\end{subfigure}
	\begin{subfigure}[t]{0.32\textwidth}
		\includegraphics[width=\textwidth]{Experiments/GrainRemoval/trg_out.png}
		\caption{Testna slika s maknutim šumom}
	\end{subfigure}
	\begin{subfigure}[t]{0.32\textwidth}
		\includegraphics[width=\textwidth]{Experiments/GrainRemoval/sp_diff_train.png}
		\caption{Razlika između dobivene i željene trening fotografije}
	\end{subfigure}
	\begin{subfigure}[t]{0.32\textwidth}
		\includegraphics[width=\textwidth]{Experiments/GrainRemoval/sp_diff_val.png}
		\caption{Razlika između dobivene i željene validacijske fotografije}
	\end{subfigure}
	\begin{subfigure}[t]{0.32\textwidth}
		\includegraphics[width=\textwidth]{Experiments/GrainRemoval/sp_diff_test.png}
		\caption{Razlika između dobivene i željene testne fotografije}
	\end{subfigure}
	\label{fig:sp_result_grid}
\end{figure}
